\documentclass{article}
\usepackage[utf8]{inputenc}
\usepackage{geometry}
\geometry{a4paper, margin=1in}
\usepackage{enumitem}

\title{Relazione Sprint 0}
\author{}
\date{}

\begin{document}

\maketitle

\section{Introduzione}
Lo Sprint zero è una fase fondamentale della creazione del progetto, dedicata a una prima pianificazione del prodotto prima di iniziare gli sprint di sviluppo veri e propri.

\section{Ruoli}
I ruoli ricoperti dal team sono:
\begin{itemize}[label={--}]
    \item \textbf{Scrum Master:} Cristian Orsi
    \item \textbf{Product Owner:} Enis Brajevic
    \item \textbf{Developer 1:} Matteo Fornaini
    \item \textbf{Developer 2:} Mattia Ferrarini
    \item \textbf{Developer 3:} Enrico Mazzotti
    \item \textbf{Developer 4:} Lorenzo Giarrusso
\end{itemize}

\section{Attività di Team Building}
Durante lo Sprint zero sono state organizzate due attività di team building attraverso i due giochi: “Scrumble” e “Escape the Boom”. Queste attività hanno permesso ai membri del team di conoscersi meglio, migliorando collaborazione, intesa e comunicazione. Le attività hanno anche portato a creare un ambiente di sviluppo positivo, fondamentale per gli sprint successivi. Concluse le attività, è stata compilata la tabella di autovalutazione del team e una breve autovalutazione più discorsiva.

\section{Backlog di User Stories}
Il Product Owner si è occupato di scrivere, commentare nel dettaglio e assegnare una priorità alle user stories sul backlog. Successivamente, con l’aiuto del team di sviluppo, sono state controllate e comprese e solo dopo “pesate” con l’assegnamento di punti che stimano la user story in termini di complessità e fattibilità tecnica.

\section{Architettura del Prodotto}
Attraverso una panoramica, è stata descritta l’architettura del prodotto, definendo i componenti principali di rispettivamente Front-End, Back-End e Database e le rispettive interazioni tra loro.

\section{Schema UML}
Sono stati creati due schemi UML: un diagramma di Deployment e un diagramma dei Use Case. Il diagramma di Deployment rappresenta l’architettura fisica del sistema, mostrando come i componenti vengono distribuiti su Front-End, Back-End e Database; mentre nel diagramma dei Use Case sono state descritte le interazioni tra gli utenti e il sistema, evidenziando le funzionalità offerte dal software. Questo processo aiuta a comprendere il dominio del problema e le interazioni tra i diversi componenti.

\section{Mockup della Grafica}
Sono stati realizzati mockup della grafica per visualizzare diverse interfacce presenti nel prodotto. Questi serviranno per raccogliere feedback iniziali dagli stakeholder e per garantire che le aspettative visive fossero allineate con le funzionalità previste.

\section{Analisi dei Rischi e Fattibilità}
Durante lo Sprint zero, è stata condotta un'analisi dei rischi e della fattibilità del progetto. Questa fase ha coinvolto l'identificazione dei potenziali ostacoli che si potrebbero presentare durante la realizzazione, come rischi tecnici, di mercato e di risorse. Parallelamente, è stata analizzata la fattibilità del progetto in termini di tempo, costi e risorse disponibili, garantendo che gli obiettivi fossero realistici e raggiungibili. Questo processo ha permesso di fornire una visione chiara della pianificazione futura per garantire il successo del progetto.


\section{Conclusione}
In conclusione, lo Sprint zero ha fornito una base solida per il progetto, con un focus sulla creazione di un backlog prioritario, preparazione delle architetture tecniche, sulla definizione dei ruoli e sul conoscimento del team.

\end{document}



