\documentclass{article}
\usepackage[utf8]{inputenc}
\usepackage{amsmath}

\title{Retrospettiva Sprint 2}
\author{}
\date{}

\begin{document}

\maketitle

\section*{Introduzione}
Nel corso della retrospettiva dello Sprint 2, sono stati analizzati diversi aspetti del lavoro svolto, con l’obiettivo di migliorare l'efficienza e la qualità del nostro lavoro per il prossimo Sprint. Sono emersi alcuni problemi che verranno corretti, per ottimizzare ulteriormente il nostro lavoro al progetto. A seguito delle discussioni durante la retrospettiva si sono scelti queste modalità per corregere questi errori.

\section*{1. Maggiore confronto tra il Product Owner e gli Stakeholder}
Durante lo Sprint 1/2, si è notato che il Product Owner non ha avuto sufficienti momenti di confronto con gli stakeholder. Questo ha causato alcune incertezze e disallineamenti nelle aspettative del prodotto. Per affrontare questo problema, è stato deciso che il PO dovrà intensificare i confronti con gli stakeholder, con l’obiettivo di raccogliere feedback e garantire che la direzione del progetto risponda alle esigenze reali del cliente. Lo Scrum Master sarà responsabile di assicurarsi che questi confronti siano regolari e produttivi, affinché gli stakeholder siano sempre aggiornati sugli sviluppi e possano influire sul prodotto in qualsiasi momento.

\section*{2. Mostrare singole funzionalità delle User Stories attraverso clip video}
Un altro aspetto discusso riguarda la comunicazione delle funzionalità implementate nelle User Stories. È emerso che, durante il primo Sprint, in particolare durante la review di fine sprint non si è tenuta traccia tangibile, come ad esempio una clip video, delle funzionalità implementate durante questa prima fase. Per migliorare la comprensione delle singole funzionalità, è stato deciso di creare delle brevi clip video che mostrano l'implementazione di ciascuna user story implementata. Questi video saranno condivisi con gli stakeholder, per permettergli di vedere effettivamente con i loro occhi cosa è stato fatto. Lo Scrum Master garantirà che i video siano realizzati con sufficiente chiarezza e che vengano utilizzati come strumento di comunicazione per favorire la trasparenza con i clienti del prodotto.

\section*{Conclusioni}
In sintesi, i punti emersi durante la retrospettiva riguardano un maggior coinvolgimento dei stakeholder, sia da una parte di sviluppo del prodotto, sia da una parte di aggiornamento periodico tramite prove, principalmente visive, del prodotto aggiornato. Sarà cura dello Scrum Master vigilare affinch´e queste azioni correttive vengano implementate e seguite con costanza per migliorare le performance del team e raggiungere gli obiettivi prefissati.

\end{document}
