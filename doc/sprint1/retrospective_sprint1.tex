\documentclass[a4paper,12pt]{article}
\usepackage[utf8]{inputenc}
\usepackage{amsmath}

\title{Retrospettiva Sprint 1}
\author{}
\date{}

\begin{document}

\maketitle

\section*{Introduzione}
Nel corso della retrospettiva dello Sprint 1, sono emersi alcuni punti che necessitano di correzione e miglioramento per ottimizzare l'efficacia del nostro lavoro nel prossimo Sprint. A seguito delle discussioni durante la retrospettiva sono stati scelti tre ambiti di intervento.



\subsection*{1. Daily Scrum su Mattermost}
È emerso che durante lo Sprint 1 la comunicazione quotidiana attraverso il canale Mattermost non è stata efficace come previsto. Alcuni membri del team non hanno partecipato attivamente, compromettendo la visibilità sugli avanzamenti e le eventuali difficoltà. Per risolvere questo problema, è stato deciso che d'ora in avanti ogni membro del team dovrà garantire la propria partecipazione al Daily Scrum su Mattermost, contribuendo tempestivamente con il proprio aggiornamento. Lo Scrum Master sarà responsabile di monitorare la regolarità delle partecipazioni e assicurerà che il canale venga utilizzato correttamente per migliorare la comunicazione e il coordinamento.

\subsection*{2. Maggiore confronto tra Developers e Product Owner}
Un altro punto discusso è la necessità di un maggior confronto tra i Developers e il Product Owner, al fine di allineare meglio le aspettative e le priorità durante lo Sprint. Durante il primo Sprint, alcuni task sono stati avviati senza un pieno allineamento sulle priorità o sui dettagli, con il rischio di lavorare inutilmente. Pertanto, si è deciso di incrementare la collaborazione tra il team di sviluppo e il PO, creando un canale su mattermost dedicato ad eventuali domande tra developers e PO per chiarire eventuali dubbi e garantire che gli obiettivi siano condivisi e ben compresi. Lo Scrum Master monitorerà la qualità e la frequenza di questi confronti.

\subsection*{3. Assegnazione delle task durante lo Sprint}
Durante lo Sprint 1, le task non sono state assegnate preventivamente a ciascun membro del team, il che ha causato difficoltà nella gestione e suddivisione del lavoro. L'assenza di una distribuzione chiara delle attività sin dall'inizio ha portato a disagi nel completamento delle task e a una mancanza di chiarezza nelle responsabilità individuali. Per evitare questi problemi, è stato deciso di assegnare in anticipo le task a ciascun developer, in modo che ognuno abbia una visione chiara delle proprie responsabilità fin dall'inizio dello Sprint. Lo Scrum Master garantirà che l'assegnazione delle task avvenga in modo puntuale e che ogni membro del team sappia esattamente su cosa concentrarsi fin dal primo giorno dello Sprint.

\section*{Conclusioni}
In sintesi, i punti emersi durante la retrospettiva riguardano la necessità di migliorare la comunicazione attraverso Mattermost, intensificare il confronto tra il team di sviluppo e il Product Owner, e ottimizzare il processo di assegnazione delle task. Sarà cura dello Scrum Master vigilare affinché queste azioni correttive vengano implementate e seguite con costanza per migliorare le performance del team e raggiungere gli obiettivi prefissati.

\end{document}

