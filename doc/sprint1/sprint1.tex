\documentclass{article}
\usepackage[utf8]{inputenc}
\usepackage{amsmath}
\usepackage{enumitem}

\title{Sprint 1}
\author{}
\date{}

\begin{document}

\maketitle

\section*{DOR}
La task è considerata "ready" quando:
\begin{itemize}[leftmargin=*]
    \item ha un obiettivo e una descrizione chiara a tutto il team.
    \item il team ne ha discusso e ha chiesto al PO chiarimenti, qualora ce ne fosse bisogno.
    \item è stata messa sul backlog in ordine di priorità e l'effort di sviluppo è stato stimato in punti, con l'intero team d'accordo.
    \item la task è completabile in uno sprint.
\end{itemize}

\section*{DOD}
La task è considerata "done" quando:
\begin{itemize}[leftmargin=*]
    \item La funzionalità prevista dalla task è stata sviluppata nel suo intero (tutti i requisiti sono stati implementati).
    \item Il codice non presenta errori di linting o warnings.
    \item Il codice è stato revisionato da almeno un altro developer e approvato.
    \item Il codice presenta commenti utili ad esplicarne il comportamento, dove necessario.
    \item I casi limite sono stati gestiti.
    \item Gli unit test passano e hanno un coverage di almeno l'80\% della nuova funzionalità.
    \item è avvenuto il commit, push, e se necessario anche il merge, delle nuove modifiche su Gitlab.
    \item è stata scritta una documentazione minimale a riguardo.
    \item l'implementazione della task è stata approvata dal PO.
\end{itemize}

\end{document}