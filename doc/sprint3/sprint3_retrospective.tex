\documentclass[a4paper,12pt]{article}
\usepackage[utf8]{inputenc}
\usepackage{lmodern}
\usepackage{titlesec}
\usepackage{hyperref}

% Custom section format
\titleformat{\section}{\large\bfseries}{\thesection.}{1em}{}
\titleformat{\subsection}{\normalsize\bfseries}{\thesubsection.}{1em}{}

\begin{document}

\section*{Retrospettiva Sprint 3}

Il completamento dello Sprint 3 ha segnato la conclusione del prodotto, ma ha evidenziato alcune problematiche che devono essere analizzate e risolte per migliorare l’efficacia del processo. Di seguito si riportano i problemi emersi e i comportamenti che lo Scrum Master dovrebbe promuovere per evitarli in futuro:

\subsection*{1. \textbf{Dettagli Persi Durante lo Sprint Planning}}
Durante lo sprint planning, non è stata dedicata sufficiente attenzione alla strutturazione delle User Stories (US) e alla loro suddivisione in task, portando a una mancanza di chiarezza e a potenziali ritardi.
Sarà cura dello Scrum Master assicurarsi che il team comprenda appieno ciascuna US prima di accettarla nello sprint backlog, utilizzando anche checklist o template standard per verificare che ogni US sia sufficientemente dettagliata e pronta per essere pianificata.

\subsection*{2. \textbf{Sottostimato il Tempo di Lavoro}}
Il team ha sottostimato il tempo necessario per completare alcune attività, portando a un sovraccarico di lavoro e all’obbligo di lavorare più del previsto.
Lo Scrum Master farà in modo di far considerare al team tutti i fattori che possono influire sui tempi di consegna, inclusi imprevisti e dipendenze esterne, inoltre verrà tenuto in conto anche un cuscinetto di tempo in caso di emergenze o imprevisti.


\subsection*{3. \textbf{Tendenza ad Aggiungere Funzionalità Non Richieste}}
In generale i membri del team hanno dedicato tempo a progettare o implementare funzionalità non previste dalle specifiche iniziali, distogliendo risorse dalle attività prioritarie.
 Lo Scrum Master, con l’aiuto del PO, deve ribadire l’importanza di attenersi agli obiettivi definiti nello sprint backlog e prevenire comportamenti di overengineering.
 Introdurre l’abitudine di verificare costantemente se le attività in corso sono allineate ai requisiti del Product Owner.

\subsection*{Conclusione}
L’ultimo sprint, pur avendo raggiunto gli obiettivi di progetto, ha evidenziato aree di miglioramento che lo Scrum Master deve affrontare per garantire una gestione più efficace del team negli sprint futuri. Attraverso una migliore preparazione delle US, stime più accurate, maggiore attenzione alla pianificazione e un rigido rispetto delle priorità definite, il team potrà evitare le difficoltà riscontrate e migliorare le proprie prestazioni.

\end{document}
